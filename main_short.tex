\documentclass[USenglish, mathserif]{beamer}


\usepackage[utf8]{inputenx} % For æ, ø, å
\usepackage{csquotes}       % Quotation marks
\usepackage{microtype}      % Improved typography
\usepackage{amssymb}        % Mathematical symbols
\usepackage{mathtools}      % Mathematical symbols
\usepackage[absolute, overlay]{textpos} % Arbitrary placement
\usepackage{media9}
\usepackage{movie15}
\usepackage{xmpmulti}
\usepackage{animate}
\usepackage{amsmath}
\usepackage{amsfonts}
\usepackage{amssymb}
\usepackage{graphicx}
\usepackage{color}
\usepackage{bigstrut}
\usepackage{hyperref}
\hypersetup{colorlinks=true,allcolors=blue}
\usepackage{tikz}
\usepackage{mathrsfs}
\usepackage{frcursive}
\usepackage{calligra}
\usepackage[T1]{fontenc}
\usepackage{natbib}
\newcommand{\setfont}[2]{{\fontfamily{#1}\selectfont #2}}

\setlength{\TPHorizModule}{\paperwidth} % Textpos units
\setlength{\TPVertModule}{\paperheight} % Textpos units
\usepackage{tikz}
\usetikzlibrary{overlay-beamer-styles}  % Overlay effects for TikZ
\setbeamerfont{normal text}{size=\normalsize, series=\rmfamily}
\AtBeginDocument{\usebeamerfont{normal text}}
\setbeamerfont{frametitle}{series=\bfseries}
\usepackage{animate}
 

\newcommand{\pdfnewline}{\texorpdfstring{\newline}{ }} 

\usepackage{mathpazo} % Use the Palatino font by default

\usetheme{UiB}


\newcommand{\eq}{\begin{equation}}
\newcommand{\no}{\end{equation}}
\newcommand{\s}{\hspace{.5cm}}
%\newcommand{\p}{\partial}
\newcommand{\p}[2]{\f{\partial#1}{\partial#2}}
\newcommand{\f}{\frac}
\newcommand{\ol}{\bar}
\newcommand{\fr}{{\rm Fr}}
\newcommand{\ti}[1]{\tilde{#1}}

\def\blue{\textcolor{blue}}
\def\red{\textcolor{red}}
\renewcommand{\raggedright}{\leftskip=0pt \rightskip=0pt plus 0cm}
\raggedright

\usefonttheme[onlymath]{serif}

%\usepackage{lmodern} % get rid of warnings
\usepackage{caption} % improved spacing between figure and caption

\DeclareCaptionLabelSeparator{horse}{:\quad} % change according to your needs
\captionsetup{
  labelsep = horse,
  figureposition = bottom
} 


\setbeamertemplate{caption}[numbered]
\captionsetup{belowskip=-8pt}
\captionsetup{aboveskip=-0.2pt}

\numberwithin{equation}{section}
\numberwithin{figure}{section}


\author{Boyuan Yu} % Your name
\institute[MU] % Your institution as it will appear on the bottom of every slide, may be shorthand to save space
{\bigskip
\small{McGill University \\ % Your institution for the title page
\smallskip
Department of Civil Engineering and Applied Mechanics}} 
\title{Hydrodynamic Instabilities in Steep-slope Flood Simulation}
\subtitle{Literature Review Presentation}
  
  
\begin{document}
%%%%%Roadmap%%%%%
\AtBeginSubsection[]{
  \frame<beamer>{ 
    \frametitle{Outline}   

\tableofcontents[currentsection,currentsubsection,sectionstyle=show/hide,subsectionstyle=show/shaded/hide,subsubsectionstyle=show/show/shaded/shaded]}
}

\begin{frame}[allowframebreaks]
    \frametitle{Table of contents}
%    \tableofcontents[currentsection] %shaded
		\setcounter{tocdepth}{2} 
        \tableofcontents 
\end{frame}

\section{Introduction}
\subsection{Destructive Power of Tsunami and Flood}
\begin{frame}{{\fontsize{16}{16}\selectfont Destructive Power of Tsunami and Flood}}
	Qualitative description about tsunami... \cite{LeVeque2011}
	
\end{frame}

\begin{frame}[allowframebreaks]{{\fontsize{16}{16}\selectfont Destructive Power of Tsunami and Flood}}
	Simulations for real flood and tsunami events: 
	
	\red{Maybe rewrite this part using a table?}
	
	\begin{itemize}
		\item \cite{Popinet2012_tohoku}: Tohoku tsunami in 2011, SWE
		\item \cite{An2015}: Malpasset dam break in 1959 and Baeksan levee failure in 2002, SWE
		\item \cite{George2011}: Malpasset dam break in 1959, SWE
		\item \cite{Oishi2013}: Tohoku tsunami, incompressible 3D NS
		\item \cite{Popinet2011}: Indian Ocean tsunami in 2004, SWE.
		\item \cite{Popinet2015}: Tohoku tsunami in 2011, SGN
		\item \cite{Popinet2020}: Tohoku tsunami in 2011, NH multilayer model
		\item \cite{LeVeque2011}: Chile tsunami in 2010, SWE
	\end{itemize}
    \framebreak
	Description about flood and tsunami events:
	\begin{itemize}
		\item \cite{Marras2020}
		\item \cite{Xie2019}
		\item 
	\end{itemize}
\end{frame}
\subsection{Previous Studies on Roll Waves}
\begin{frame}{Intro}


\end{frame}


% item-by-item presentation
%\begin{frame}{Introduction}
%\begin{itemize}[<+-|alert@+>]
%\item The \red{erosion} of riverbed and the \red{transport} of rocks and muds by floods are processes associated with instability, waves and turbulence in shear flow on channel of
%steep slope. 
%\item The driven force to the flow is the \red{gravity}. The resistance is the \red{bed-friction}.
%\item  Chu, Wu and Khayat (CWK 1991) examined the effect using a \red{rigid-lid approximation}. {Without the waves}, bed-friction is a  {stabilizing} influence. The role of the waves was investigated recently by Karimpour \& Chu (KC 2016) \red{ignoring the effect of the bed-friction}. When bed-friction is ignored, the wave is also a {stabilizing} influence. 
%\item We study the instability in this thesis for the \red{combined} effect of
%the waves and the bed-friction.
%
%\end{itemize}
%\end{frame}
 


%\begin{figure}[h!]
%\centering
%\includegraphics[width=1\linewidth]{defsketch.jpg}
%\caption{Definition sketch of the transverse shear flow.}
%\label{defsketch1}
%\end{figure}

\section{Numerical Methods}
\subsection{Numerical Methods for SWEs}
\subsubsection{Godunov-type FVM}
\subsection{Numerical Methods for Interfacial Flow}
\subsection{CFD Software for Flood and Tsunami Modelling}


\section{Previous, Present and Future Research}
  
  
\section{References}
\begin{frame}[allowframebreaks]{References}
	\textit{Modified from the .bib file of cylinder impact draft.}
	\bibliographystyle{jfm}
	\nocite{*}
	\small\bibliography{example}
	%\bibliographystyle{alpha}
	% \tiny\bibliographystyle{alpha}
\end{frame}

%\begin{frame}[allowframebreaks]{References}
%\scriptsize
%\setbeamertemplate{bibliography item}[text]
%\bibliographystyle{plain}
%\bibliography{example}
%\end{frame}

%\begin{frame}{}
%\bigskip
%\bigskip
%\bigskip
%\bigskip
%\bigskip
%
%
%\begin{center}
% 
%\Huge \setfont{calligra}{Thanks for your attention.}
%  \end{center}
%  \bigskip
%\bigskip
%\bigskip
%\medskip
%
%
%\normalsize
%\begin{flushright}
%%    \line(1,0){135}\\
%    \textbf{Contact information:}\\
%     {Boyuan Yu}\\
%    E-mail: \texttt{boyuan.yu@mcgill.ca}\\
%%    Department of Civil Engineering and Applied Mechanics\\
%    McGill University 
%\end{flushright}
%\end{frame}



  
  
\end{document}